In this section we will present the family of algorithms we call $k$-Repetitive-Nearest-Neighbor ($k$-RNN). 
Later we will show a property about the quality of the solution found by the algorithm and present some of its variations.
Following a popular convention for TSP-related algorithms, we will also refer to a Hamiltonian cycle as a "tour". 	

Let $G = (V, E)$ be a complete Graph and $k \in \mathbb{N}$. Let $v_1, v_2, \mathellipsis, v_k$ be vertices of $G$. 
Let $c: V \times V \rightarrow \mathbb{R}$ be a cost function describing the cost of the edge between two vertices. 
The algorithm consists of the following steps:


\begin{description}
	\item[Step 1:] For every combination of $k$ distinct vertices $v_1, v_2, \mathellipsis, v_k$ create the partial tour $T = (v_1, v_2, \mathellipsis, v_k)$ and mark the vertices $v_1, v_2, \mathellipsis, v_k$ as visited.
	
	\item[Step 2:] Set $v_e = v_k$. While there are unvisited vertices left: 
	Select $v_{i+1}$ as the nearest unvisited neighbor of $v_e$. 
	If there are multiple nearest neighbors, select any.
	Mark $v_{i+1}$ as visited and set $v_e = v_{i+1}$.
	
	\item[Step 3:] Among all $k!$ tours found select the shortest as the result
\end{description}	

Note that for $k = 1$ the algorithm equals the well-known Repetitive-Nearest-Neighbor approach, and one might define the case $k = 0$ as the popular Nearest-Neighbor algorithm, where one starting node gets selected randomly.

The running time of the algorithm consists of two major parts. 
The first is the outer for-loop in Step 1. 
Since for every $k \in \mathbb{N}$ there exist $\frac{n!}{(n-k)!}$ (ordered) combinations of nodes and since $\frac{n!}{(n-k)!} \in O(n^k)$, the for-loop in Step 1 does a total of $O(n^k)$ iterations.
The second important part for the running time is Step 2. In order to find the nearest unvisited neighbor, the algorithm considers a maximum of $n$ edges emerging from the current node. 
Since this is done for $n - k = O(n)$ nodes, the total running time of Step 2 is $O(n^2)$.
Therefore a $k$-RNN run takes time $O(n^2 \cdot n^k) = O(n^{k+2})$.

Similar to the $k$-Opt algorithms \cite{CROES1958, LIN1973}, for $k = n$ $k$-RNN also gives the exact solution. In this case, the algorithm degrades to a brute-force search.
Due to the running time of $O(n!)$ this is not the preferable way to obtain it.

In the following we refer to an execution of $k$-RNN for a specific instance and fixed $k$ as a "run" of the algorithm.
We no present a property of the algorithm about the quality of the solution found by $k$-RNN.
\begin{theorem}
	For $k < l$, if there exist no nodes $a, b, c$ with $c(a, b) = c(a, c)$, the result of a $l$-RNN run is always better or equal to the result of a $k$-RNN run.
\end{theorem}
\begin{proof}
	Let $T$ be a tour found by a $k$-RNN run and $(v_1, v_2, \mathellipsis, v_l)$ the first $l$ nodes of $T$. 
	A $l$-RNN run considers every permutation of $l$ nodes as a starting tour, so especially the permutation $(p_1, p_2, \mathellipsis, p_l)$ where $p_i = v_i$ for $1 \leq i \leq l$. 
	From there on, both runs construct the same tour.
\end{proof}

There are also some variations of the presented algorithm of which we want to mention two.

In step 2.2 node $v_{i+1}$ gets selected as the nearest unvisited neighbor of $v_i$. This can be extended in such a way, that the new node can also be selected as the nearest unvisited neighbor of the current first node of the tour. 
We refer to this approach as bi-directional $k$-RNN. Step 2 of the algorithm would then look the following:

\begin{description}
	\item[Step 2:] Set $v_e = v_k$ and $v_s = v_1$. 
	While there are unvisited vertices left: 
	Select $q$ as 
	\[
	\arg \min\{c(v_e, p),\ c(v_s, p) |\ p \in V \text{ and $p$ is unvisited}\}
	\] 
	and mark $q$ as visited. If $c(v_s, q) < c (v_e, q)$, insert $q$ at the start of $T$ and set $v_s = q$. Else append $q$ to the end of $T$ and set $v_e = q$.
\end{description}

Results...

Another variant is that when you find multiple edges with the same weight in Step 2, you continue constructing multiple paths, one for each duplicate. 
Although this eliminates the potential randomness, in some cases this will make the running time exponential.
