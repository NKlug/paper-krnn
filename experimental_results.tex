In the following we are going to present some experimental results of the algorithm. The experiments where conducted for several instances of TSPLIB \cite{REINELT1995}, symmetric as well as asymmetric, comparing $1$- and $2$-RNN runs.
Because of the running time, larger $k$-values could not be tested adequately.

\begin{figure}[!htbp]
	\centering
	\begin{minipage}{\linewidth}
		\centering
		\scriptsize
		\begin{tabular}{*{3}{L{1.6cm}} >{\bfseries}L{1.6cm} L{1.6cm} >{\bfseries}L{1.6cm} L{1.6cm} >{\bfseries}L{1.6cm}}
			\toprule
			&& \multicolumn{2}{c}{1-RNN}& \multicolumn{2}{c}{2-RNN} & \multicolumn{2}{c}{Bi-2-RNN} \\
			\cmidrule(lr){3-4}
			\cmidrule(lr){5-6}
			\cmidrule(lr){7-8}
			Dataset & Optimum & Result & Excess & Result & Excess & Result & Excess \\
			\midrule
			a280      & 2579   & 2975   & 15.35  & 2953   & 14.50 & 2951   & 14.42  \\
			berlin52  & 7542   & 8181   & 8.47   & 7968   & 5.65  & 8380   & 11.11  \\
			bier127   & 118282 & 133953 & 13.25  & 128589 & 8.71  & 129133 & 9.17   \\
			brazil58  & 25395  & 27384  & 7.83   & 27213  & 7.16  & 27115  & 6.77   \\
			brg180    & 1950   & 8890   & 355.90 & 2020   & 3.59  & 8890   & 355.90 \\
			ch130     & 6110   & 7129   & 16.68  & 6903   & 12.98 & 6833   & 11.83  \\
			ch150     & 6528   & 7113   & 8.96   & 7113   & 8.96  & 7075   & 8.38   \\
			d1291     & 50801  & 58681  & 15.51  & 58681  & 15.51 & 58460  & 15.08  \\
			d1655     & 62128  & 73369  & 18.09  & 72554  & 16.78 & 71858  & 15.66  \\
			d198      & 15780  & 17620  & 11.66  & 17405  & 10.30 & 17753  & 12.50  \\
			d493      & 35002  & 40186  & 14.81  & 40186  & 14.81 & 39821  & 13.77  \\
			d657      & 48912  & 60174  & 23.03  & 59310  & 21.26 & 58874  & 20.37  \\
			dantzig42 & 699    & 864    & 23.61  & 826    & 18.17 & 848    & 21.32  \\
			eil101    & 629    & 746    & 18.60  & 743    & 18.12 & 738    & 17.33  \\
			eil51     & 426    & 482    & 13.15  & 472    & 10.80 & 483    & 13.38  \\
			eil76     & 538    & 608    & 13.01  & 598    & 11.15 & 576    & 7.06   \\
			fl1400    & 20127  & 25115  & 24.78  & 24719  & 22.82 & 24587  & 22.16  \\
			fl417     & 11861  & 13887  & 17.08  & 13866  & 16.90 & 13581  & 14.50  \\
			fri26     & 937    & 965    & 2.99   & 959    & 2.35  & 960    & 2.45   \\
			gil262    & 2378   & 2823   & 18.71  & 2767   & 16.36 & 2768   & 16.40  \\
			gr120     & 6942   & 8438   & 21.55  & 8335   & 20.07 & 8411   & 21.16  \\
			gr17      & 2085   & 2178   & 4.46   & 2178   & 4.46  & 2178   & 4.46   \\
			gr21      & 2707   & 3003   & 10.93  & 2958   & 9.27  & 2998   & 10.75  \\
			gr24      & 1272   & 1553   & 22.09  & 1400   & 10.06 & 1476   & 16.04  \\
			gr48      & 5046   & 5840   & 15.74  & 5561   & 10.21 & 5695   & 12.86  \\
			hk48      & 11461  & 12137  & 5.90   & 12031  & 4.97  & 11990  & 4.62   \\
			kroA100   & 21282  & 24698  & 16.05  & 24582  & 15.51 & 24548  & 15.35  \\
			kroA150   & 26524  & 31479  & 18.68  & 31320  & 18.08 & 31234  & 17.76  \\
			kroA200   & 29368  & 34543  & 17.62  & 34543  & 17.62 & 35329  & 20.30  \\
			kroB100   & 22141  & 25884  & 16.91  & 25255  & 14.06 & 25546  & 15.38  \\
			kroB150   & 26130  & 31611  & 20.98  & 31524  & 20.64 & 30043  & 14.98  \\
			kroB200   & 29437  & 35389  & 20.22  & 35283  & 19.86 & 35454  & 20.44  \\
			kroC100   & 20749  & 23660  & 14.03  & 23603  & 13.75 & 23970  & 15.52  \\
			kroD100   & 21294  & 24852  & 16.71  & 24603  & 15.54 & 23722  & 11.40  \\
			kroE100   & 22068  & 24782  & 12.30  & 24445  & 10.77 & 24185  & 9.59   \\
			lin105    & 14379  & 16935  & 17.78  & 16147  & 12.30 & 15878  & 10.42  \\
			lin318    & 42029  & 49201  & 17.06  & 49201  & 17.06 & 48996  & 16.58  \\
			linhp318  & 41345  & 49201  & 19.00  & 49201  & 19.00 & 48996  & 18.51  \\
			nrw1379   & 56638  & 68531  & 21.00  & 67873  & 19.84 & 67415  & 19.03  \\
			p654      & 34643  & 43027  & 24.20  & 42935  & 23.94 & 42493  & 22.66  \\
			pa561     & 2763   & 3279   & 18.68  & 3269   & 18.31 & 3284   & 18.86  \\
			pcb1173   & 56892  & 70115  & 23.24  & 69085  & 21.43 & 69325  & 21.85  \\
			pcb442    & 50778  & 58950  & 16.09  & 58682  & 15.57 & 58599  & 15.40  \\
			pr76      & 108159 & 130921 & 21.04  & 128749 & 19.04 & 129467 & 19.70  \\
			si1032    & 92650  & 94083  & 1.55   & 93981  & 1.44  & 93731  & 1.17   \\
			si175     & 21407  & 22000  & 2.77   & 21906  & 2.33  & 21927  & 2.43   \\
			si535     & 48450  & 50036  & 3.27   & 50032  & 3.27  & 49853  & 2.90   \\
			swiss42   & 1273   & 1437   & 12.88  & 1425   & 11.94 & 1350   & 6.05   \\
			\bottomrule
		\end{tabular}
	\end{minipage}
	\caption{Results for 48 instances of the Symmetric TSP taken from \cite{REINELT1995}. The optimum and the result are given in absolute values. The excess represents the percentage by which the result exceeds the optimum.}
	\label{fig:STSP}
\end{figure}


Figure \ref{fig:STSP} shows the results of experiments conducted for 48 instances of the STSP taken from TSPLIB \cite{REINELT1995}.
It can be observed that the results of $2$-RNN are only slightly better than those of $1$-RNN, sometimes they are even equal.
We would like to point out two specific instances, the first is \texttt{brg180}.
For this 180-city instance, standard 1-RNN returns the poor result of 8890 (355.9\% above the optimum) while 2-RNN is able to avoid this returning a tour with length 2020 (3.59\% above the optimum).
For one other instance (\texttt{br17}), $2$-RNN even produces the exact result.

In general all results produced by $2$-RNN are reasonable. 
On average, the tours produced are 10 \% to 40 \% longer than the optimum.
Using these observations, we expect the algorithm to perform similarly for other instances although there might be instances were the output is of unreasonable quality.

Considering that a $2$-RNN run for an instance with 1379 vertices (namely TSPLIB's \texttt{nrw1379}) takes about 60 minutes to finish while a $1$-RNN run for the same instance only takes about 3 seconds (tested on a standard laptop), the quality of the solution does not increase by a big enough amount to justify the running time.
In general, this makes $2$-RNN less preferable than $1$-RNN.

The results of the bidirectional variant of 2-RNN are of different quality. For some instances they outperform $2$-RNN, for some others $2$-RNN performs better.
As the bidirectional version tries to extend the tour in both directions rather than only one, the running time is about twice as long as for the normal variant.

\begin{figure}[ht]
	\centering
	\begin{minipage}{\linewidth}
		\centering
		\begin{tabular*}{\textwidth}{l||r|| l | l || l | l }
			Dataset & Optimum & 1-RNN & Excess (in \%) & 2-RNN & Excess (in \%) \\
			\hline\hline
			br17    & 39    & 56    & 43.59 & 39    & 0.00  \\
			ft53    & 6905  & 8584  & 24.32 & 8323  & 20.54 \\
			ft70    & 38673 & 41815 & 8.12  & 41633 & 7.65  \\
			ftv33   & 1286  & 1590  & 23.64 & 1544  & 20.06 \\
			ftv35   & 1473  & 1667  & 13.17 & 1606  & 9.03  \\
			ftv38   & 1530  & 1759  & 14.97 & 1709  & 11.70 \\
			ftv44   & 1613  & 1844  & 14.32 & 1829  & 13.39 \\
			ftv47   & 1776  & 2173  & 22.35 & 2149  & 21.00 \\
			ftv55   & 1608  & 1948  & 21.14 & 1854  & 15.30 \\
			ftv64   & 1839  & 2202  & 19.74 & 2202  & 19.74 \\
			ftv70   & 1950  & 2287  & 17.28 & 2218  & 13.74 \\
			ftv170  & 2755  & 3582  & 30.02 & 3559  & 29.18 \\
			kro124p & 26230 & 43316 & 65.14 & 43102 & 64.32 \\
			p43     & 5620  & 5684  & 1.14  & 5653  & 0.59  \\
			rbg323  & 1326  & 1702  & 28.36 & 1684  & 27.00 \\
			rbg358  & 1163  & 1747  & 50.21 & 1719  & 47.81 \\
			rbg403  & 2465  & 3497  & 41.87 & 3460  & 40.37 \\
			ry48p   & 14422 & 15575 & 7.99  & 15575 & 7.99  \\
		\end{tabular*}
	\end{minipage}
	
	\caption{Results for XX asymmetric instances of the TSP taken from \cite{REINELT1995}}
	\label{fig:table}
\end{figure}

Figure \ref{fig:ATSP} shows experimental results for 18 instances of the ATSP.
In comparison to the $2$-RNN results for the STSP, the $2$-RNN results here seem to be slightly worse:
Whereas there is no instance where the tour length exceeds the optimum by more than 25\% for STSP there are 4 instances for the ATSP where this is the case.
